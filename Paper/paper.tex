\documentclass[conference]{IEEEtran}
\IEEEoverridecommandlockouts

\usepackage{cite}               % IEEE numeric citations
\usepackage{amsmath,amssymb,amsfonts}
\usepackage{algorithmic}
\usepackage{graphicx}
\usepackage{textcomp}
\usepackage{xcolor}
\usepackage{tikz}
\usetikzlibrary{positioning, fit, arrows.meta, shapes}
\usetikzlibrary{shapes.geometric, arrows, positioning}

\newcommand{\empt}[2]{$#1^{\langle #2 \rangle}$}



\def\BibTeX{{\rm B\kern-.05em{\sc i\kern-.025em b}\kern-.08em
    T\kern-.1667em\lower.7ex\hbox{E}\kern-.125emX}}

\begin{document}

\title{Transformation of Jago Bank Business Model  \\
\thanks{Identify applicable funding agency here. If none, delete this.}
}

\author{\IEEEauthorblockN{1\textsuperscript{st} Samuel Mualatua Jeremy N}
	\IEEEauthorblockA{\textit{dept. name of organization (of Aff.)} \\
		\textit{name of organization (of Aff.)}\\
		City, Country \\
		email address or ORCID}
	\and
    \IEEEauthorblockN{2\textsuperscript{nd} Jose Galbraith Hasintongan}
	\IEEEauthorblockA{\textit{dept. name of organization (of Aff.)} \\
		\textit{name of organization (of Aff.)}\\
		City, Country \\
		email address or ORCID}
	\and
	\IEEEauthorblockN{3\textsuperscript{rd} Muhammad Firdaus}
	\IEEEauthorblockA{\textit{dept. name of organization (of Aff.)} \\
		\textit{name of organization (of Aff.)}\\
		City, Country \\
		email address or ORCID}
	\and
	\IEEEauthorblockN{4\textsuperscript{th} Jon Felix Germinian}
	\IEEEauthorblockA{\textit{dept. name of organization (of Aff.)} \\
		\textit{name of organization (of Aff.)}\\
		City, Country \\
		email address or ORCID}
}


\maketitle

\begin{abstract}
This paper examines how information technology (IT) has transformed the business model of PT Bank Jago Tbk, which was previously known as Bank Artos. Following its acquisition, the former Bank Artos was redesigned and relaunched as Bank Jago, evolving from a small traditional bank into a fully digital, ecosystem-based bank. Using a descriptive qualitative approach, this study analyzes how IT has changed the bank’s value creation, customer acquisition, and revenue structure. The paper also compares Bank Jago with similar companies in Indonesia and abroad, including ********.
\end{abstract}

\begin{IEEEkeywords}
Bank Jago, Bank Artos, digital transformation, business model
\end{IEEEkeywords}

\section{Introduction}

Digital transformation has become a necessity in the banking sector due to rising public expectations for services that are fast, efficient, safe, and secure, as well as financing and services that can be conducted at any location \cite{Linggadjaya2022DigitalTransformation}.

Bank Jago was originally established as Bank Artos, a conventional bank created in 1992 and later listed on the stock exchange in 2016. A major turning point occurred in 2019 when Bank Artos was acquired by Metamorfosis Ekosistem Indonesia and Wealth Track Technology, becoming its new controlling shareholders. Following this acquisition, on June 11, 2020, the company name was officially changed to Bank Jago.

The rise of digital competition in Indonesia further strengthens the relevance of examining Bank Jago’s transformation. Advances in financial technology have intensified industry rivalry, with fintech firms offering more convenience, lower costs, and broader access than traditional banks \cite{Khuan2022Competitiveness}. Digital banking transactions in Indonesia continue to increase each year, reflecting a nationwide shift toward simpler and faster digital financial services \cite{Khuan2022Competitiveness}. Within this landscape, Bank Jago represents a significant example of deep strategic transformation, having made substantial investments in technology and infrastructure to reinvent itself as a fully digital, ecosystem-based bank \cite{Khuan2022Competitiveness}. This makes Bank Jago a relevant case for understanding how a legacy institution can reposition its business model in response to digital disruption.

\section{Literature Review}


\subsection{Digital Transformation in Banking }

The global banking sector is undergoing a shift in the digital era. The advent of information and communication technology has forced a departure from traditional, branch-based banking models toward integrated, digital-first ecosystems \cite{osei2023unlocking}. This digital shift affects all aspects of the baking industry, from back-office operations and risk management to customer engagement and service delivery.

The banking industry, once a bastion of tradition, now faces disruptive pressures from multiple fronts:

\begin{enumerate}
    \item Evolving Customer Expectations: Customers of the modern era now demand faster, easier, and more connected banking experiences through their smartphones and online platforms
    \item The FinTech Challenge: FinTech firms and neobanks, leverage agile technology and customer-centric models. Nearly 90\% of banks fear losing business to these agile competitors \cite{osei2023unlocking}. These FinTech companies and new digital banks have become significant rivals in areas like digital payments and peer-to-peer lending \cite{alqararah2025role}
\end{enumerate}

For traditional banks, digital transformation is an essential journey to stay competitive in the industry \cite{alqararah2025role}. This change isn't just about making things digital; it's about completely changing how the bank operates and how it makes money \cite{alqararah2025role}.


\section{Methodology}

% This paper uses a qualitative method based on existing literature to examine how Bank Rakyat Indonesia (BRI) has changed its business model through its BRIAPI open banking initiative. The analysis focuses on how open banking influences BRI’s value proposition, key activities, and revenue streams. To strengthen the analysis, the study includes comparative observations with similar initiatives undertaken by other banks.

% \subsection{Data Collection Method}

% The study uses only secondary data from publicly accessible documents. The data will be collected from these sources:

% \begin{enumerate}
%     \item Academic Literature: Articles published in journals that focus on digital transformation in the banking industry.
%     \item Financial News: Articles from reliable financial news websites.
% \end{enumerate}

% \subsection{Data Analysis Technique}

% The data was analyzed using two main methods:

% \begin{enumerate}
%     \item Descriptive Analysis: This method involved combining information from the different sources to create a comprehensive overview of Jago Bank digital transformation. 
%     \item Comparative Analysis: 
% \end{enumerate}

% ---------- Bibliography ----------
\bibliographystyle{IEEEtran}
\bibliography{references}  % references.biba must be in same folder

\end{document}
