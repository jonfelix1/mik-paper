\documentclass[conference]{IEEEtran}
\IEEEoverridecommandlockouts

\usepackage{cite}               % IEEE numeric citations
\usepackage{amsmath,amssymb,amsfonts}
\usepackage{algorithmic}
\usepackage{graphicx}
\usepackage{textcomp}
\usepackage{xcolor}
\usepackage{tikz}
\usetikzlibrary{positioning, fit, arrows.meta, shapes}
\usetikzlibrary{shapes.geometric, arrows, positioning}

\newcommand{\empt}[2]{$#1^{\langle #2 \rangle}$}



\def\BibTeX{{\rm B\kern-.05em{\sc i\kern-.025em b}\kern-.08em
    T\kern-.1667em\lower.7ex\hbox{E}\kern-.125emX}}

\begin{document}

\title{Business Model Transformation of Bank Artos to Bank Jago  \\
\thanks{Identify applicable funding agency here. If none, delete this.}
}

\author{\IEEEauthorblockN{1\textsuperscript{st} Jon Felix Germinian}
	\IEEEauthorblockA{\textit{Faculty of Computer Science} \\
		\textit{Universitas Indonesia}\\
		Jakarta, Indonesia \\
		email address or ORCID}
	\and
    \IEEEauthorblockN{2\textsuperscript{nd} Jose Galbraith Hasintongan}
	\IEEEauthorblockA{\textit{Faculty of Computer Science} \\
		\textit{Universitas Indonesia}\\
		Jakarta, Indonesia \\
		email address or ORCID}
	\and
	\IEEEauthorblockN{3\textsuperscript{rd} Muhammad Firdaus}
	\IEEEauthorblockA{\textit{Faculty of Computer Science} \\
		\textit{Universitas Indonesia}\\
		Jakarta, Indonesia \\
		email address or ORCID}
	\and
	\IEEEauthorblockN{4\textsuperscript{th} Samuel Mualatua Jeremy N}
	\IEEEauthorblockA{\textit{Faculty of Computer Science} \\
		\textit{Universitas Indonesia}\\
		Jakarta, Indonesia \\
		email address or ORCID}
}


\maketitle

\begin{abstract}
	This paper analyzes the business model transformation of Bank Artos into Bank Jago, examining how Information Technology shifted its strategic positioning from a traditional commercial bank to a digital ecosystem-based bank.
	The study employs a qualitative descriptive approach with a single-case study design. Data was triangulated from annual reports, financial statements, and technical documentation from 2018 to 2024. The analysis utilizes Applegate’s Business Model Framework and Savage’s Management Generations to categorize strategic shifts.
	The findings indicate that IT enabled Bank Jago to transition from a rigid Generation 3 hierarchy to an agile Generation 5 network, fundamentally altering its cost structure and reducing its BOPO ratio from 258\% to sustainable levels. Unlike competitors that focus on standalone digital apps, Bank Jago’s transformation is distinct in its "embedded finance" strategy, leveraging deep integration with the GoTo ecosystem to acquire customers, differentiating it from the direct-to-consumer models of Bank Jenius and KakaoBank.
\end{abstract}

\begin{IEEEkeywords}
	Bank Jago, Bank Artos, digital transformation, business model, embedded finance
\end{IEEEkeywords}

\section{Introduction}

Digital transformation has become a necessity in the banking sector due to rising public expectations for services that are fast, efficient, safe, and secure, as well as financing and services that can be conducted at any location \cite{Linggadjaya2022DigitalTransformation}.

Bank Jago was originally established as Bank Artos, a conventional bank created in 1992 and later listed on the stock exchange until 2016. A major turning point occurred in 2019 when Bank Artos was acquired by Metamorfosis Ekosistem Indonesia and Wealth Track Technology, becoming its new controlling shareholders. Following this acquisition, on June 11, 2020, the company name was officially changed to Bank Jago.

The rise of digital competition in Indonesia further strengthens the relevance of examining Bank Jago’s transformation. Advances in financial technology have intensified industry rivalry, with fintech firms offering more convenience, lower costs, and broader access than traditional banks \cite{Khuan2022Competitiveness}. Digital banking transactions in Indonesia continue to increase each year, reflecting a nationwide shift toward simpler and faster digital financial services \cite{Khuan2022Competitiveness}. Within this landscape, Bank Jago represents a significant example of deep strategic transformation, having made substantial investments in technology and infrastructure to reinvent itself as a fully digital, ecosystem-based bank \cite{Khuan2022Competitiveness}. This makes Bank Jago a relevant case for understanding how a legacy institution can reposition its business model in response to digital disruption.

\section{Literature Review}

\subsection{Business Model Framework}

According to \cite{Applegate2009} a business model is a complex system comprising three interlocking components of strategy, capability, and value. This framework serves as the primary tool for this case study to categorize the changes driven by IT.

\begin{enumerate}
	\item Strategy: This component defines the organization's market positioning, product positioning, and business network positioning. In the context of digital banking, IT enables a shift from "product-push" strategies to "customer-pull" strategies, allowing for personalization and the servicing of previously unprofitable segments (e.g., the unbanked) through low-cost digital channels.
	\item Capabilities: This refers to the resources (people, processes, and infrastructure) required to execute the strategy. \cite{Applegate2009} emphasizes that IT is no longer just support; it is the infrastructure itself. For Bank Jago, this involves the transition from legacy mainframes to cloud-native architectures (Mambu/Google Cloud), which fundamentally alters the cost structure from Capital Expenditure (CapEx) to Operational Expenditure (OpEx).
	\item Value: This component analyzes the economic logic of how value is created for stakeholders. IT changes the "profit formula" by reducing transaction costs and Customer Acquisition Costs (CAC) through ecosystem integration rather than physical branch expansion. 
\end{enumerate}

\subsection{Digital Transformation in Banking }

The global banking sector is undergoing a shift in the digital era. The advent of information and communication technology has forced a departure from traditional, branch-based banking models toward integrated, digital-first ecosystems \cite{osei2023unlocking}. This digital shift affects all aspects of the baking industry, from back-office operations and risk management to customer engagement and service delivery.

The banking industry, once a bastion of tradition, now faces disruptive pressures from multiple fronts:

\begin{enumerate}
    \item Evolving Customer Expectations: Customers of the modern era now demand faster, easier, and more connected banking experiences through their smartphones and online platforms
    \item The FinTech Challenge: FinTech firms and neobanks, leverage agile technology and customer-centric models. Nearly 90\% of banks fear losing business to these agile competitors \cite{osei2023unlocking}. These FinTech companies and new digital banks have become significant rivals in areas like digital payments and peer-to-peer lending \cite{alqararah2025role}
\end{enumerate}

For traditional banks, digital transformation is an essential journey to stay competitive in the industry \cite{alqararah2025role}. This trend is changing how the bank operates and how they make money \cite{alqararah2025role}.


\subsection{The Ecosystem Driver and Embedded Finance}

\cite{Weill2018} introduced the concept of Embedded Finance and the Ecosystem Driver model. In this model, banks no longer act as standalone destinations but embed their financial services (payments, lending, savings) directly into the customer’s daily digital activities (e.g., ride-hailing, e-commerce).  

This aligns with the shift from "Pipeline" business models (linear value creation) to "Platform" business models (networked value creation). Research by \cite{Linggadjaya2022DigitalTransformation} specifically on Bank Jago highlights that digital transformation requires a "digital culture" and "collaboration" capabilities, which allows the bank to leverage external data for credit scoring, a key differentiator from conventional models.   

\subsection{The Evolutionary Trajectory of Management Generations}

\cite{Savage1996} framework divides the progression of organizational maturity to five distinct generations. Each generation represents a fundamental shift in how people, technology, and resources are combined to create value. The theoretical argument is that while technology has advanced to the Fifth Generation (parallel processing/networking), most organizations remain trapped in Second or Third Generation management structures (hierarchies/matrix), creating a "organizational lag" that stifles digital transformation.

\begin{itemize}
	\item Generation 1: Proprietorship (The Craft Era).
	Characterized by the owner-entrepreneur who performs all functions. Information flow is informal and contained within the mind of the proprietor, limiting scalability.
	pasi
	\item Generation 2: Steep Hierarchies (The Industrial Era).
	Based on the division of labor and "Command and Control" logic. Functional departments (e.g., Finance, Ops) operate in silos with vertical information flow, prioritizing stability over agility.
	
	\item Generation 3: The Matrix (Electronic Data Processing)
	The introduction of IT to automate manual tasks within silos. While data is digitized, it creates "Islands of Automation" where systems do not communicate. This represents the legacy Bank Artos model: rigid and bureaucratically fragmented.
	
	\item Generation 4: Computer Interfacing (Technical Integration).
	Organizations establish technical connectivity (systems talking to systems) but retain cultural silos. The network exists, but the management style remains hierarchical.
	
	\item Generation 5: Human Networking (The Knowledge Era).
	Gen 5 (Knowledge/Teaming): Characterized by cross-functional agility and real-time connectivity. Bank Jago’s adoption of "Life-Centric" financial solutions and agile "Squad" organizational structures represents a move toward Generation 5, where IT enables the organization to function as a dynamic network of partners rather than a rigid hierarchy.   
\end{itemize}



\section{Methodology}

\subsection{Research Design}

This study uses a single-case study methodology with a qualitative descriptive research design. The case study approach was chosen because it enables a thorough examination of the transformation of digital banking in the actual setting of Bank Jago. This method works especially well for addressing the "how" and "why" of how IT alters Bank Jago's business strategy. 

\subsection{Data Collection}

Data was collected using secondary data analysis techniques. To ensure the validity and reliability of the research, data was triangulated from multiple official and academic sources covering the period 2018 (Pre-acquisition/Bank Artos) to 2024 (Post-transformation/Bank Jago):


\begin{enumerate}
	\item Corporate Filings: Annual Reports (2019–2023), Sustainability Reports, and Financial Statements published on the Indonesia Stock Exchange (IDX) and the Bank Jago investor relations website. Key focus areas included the Management Discussion and Analysis (MD\&A) sections to extract strategic intent.
	
	\item Technology Case Studies: Technical documentation and case studies published by Bank Jago’s technology partners, specifically Google Cloud Platform (GCP) and Mambu. These documents provide the technical evidence of infrastructure changes (e.g., cloud migration, API architecture).
	
	\item Academic Literature: Peer-reviewed journal articles discussing Bank Jago’s competitiveness, digital transformation strategies, and governance challenges ,.   
\end{enumerate}


\subsection{Data Analysis Technique}

The collected data was analyzed using content analysis and comparative analysis based on the \cite{Applegate2009} Framework components:

\begin{enumerate}
	\item     Strategic Comparison: Contrasting Bank Artos’s 2018 market positioning (conventional, branch-heavy) with Bank Jago’s 2024 ecosystem positioning (embedded, branchless).
	
	\item Capability Assessment: Mapping the shift in IT infrastructure (Legacy Core vs. Cloud Native) and Organizational Structure (Functional vs. Agile Squads).
	
	\item Economic Logic Evaluation: Comparing key financial ratios such as Cost-to-Income Ratio (CIR), Net Interest Margin (NIM), and Customer Acquisition Cost (CAC) to determine the economic impact of the business model change.
\end{enumerate}

\section{Discussion and Analysis}

\subsection{Legacy state of presentasinBank Artos}

For the majority of its existence, Bank Artos operated as a traditional commercial bank focused on the Bandung and Jakarta areas. Its business model was a classic "Pipeline" structure: it gathered deposits from a small customer base and channeled them into loans. This model relies heavily on physical branches and personal relationships, which naturally limits scalability \cite{Haryanto2022}.

By 2018, the limitations of this strategy were evident. The macro-economic environment in Indonesia was shifting towards digital adoption, yet Bank Artos remained tethered to its traditional roots. The bank's annual reports \cite{jago2018annual} from this period revealed a defensive strategy focused on survival rather than growth. The bank was classified as BUKU I (the lowest capital category), which legally restricted the range of products and services it could offer, further hampering its ability to compete with larger BUKU III and IV banks.

The most telling indicator of Bank Artos's financial distress was its Operating Expenses to Operating Income (BOPO) ratio. A BOPO ratio above 100\% denotes that the bank was operating at a loss, with expenses exceeding generated revenue.

In 2018, Bank Artos recorded a BOPO of 118.02\%.  By 2019, this deteriorated further to 258.09\%. This indicates that for every 1 Rupiah earned, the bank was spending nearly 2.6 Rupiah. Full metric Bank Artos condition can be seen in Table \ref{tab:financial-decomp-artos}

\begin{table}[ht]
	\centering
	\caption{Financial Decomposition of Bank Artos (Pre-Transformation)}
	\label{tab:financial-decomp-artos}
	\begin{tabular}{|l|c|c|c|}
		\hline
		\textbf{Metric} & \textbf{2017} \cite{jago2017annual} & \textbf{2018} \cite{jago2018annual} & \textbf{2019} \cite{jago2019annual} \\ 
		\hline
		Net Profit/Loss (Loss) & (IDR 10.2 B) & — & (IDR 165 B) \\ \hline
		BOPO & 113.70\% & 118.02\% & 258.09\% \\ \hline
		NPL (Net) & Low & 4.49\% & 0.05\%* \\ \hline
		LDR & 76.74\% & 73.42\% & 52.80\% \\ \hline
		NIM & 4.84\% & 4.87\% & 2.05\% \\ \hline
	\end{tabular}
\end{table}

 

\subsection{Designing Bank Jago}

The transformation began in earnest with the acquisition of Bank Artos in 2019 by PT Metamorfosis Ekosistem Indonesia (MEI) and Wealth Track Technology (WTT).

The new management introduced a "Life-Centric" philosophy, rejecting the traditional bank-centric view of the world. Jerry Ng, drawing on his experience transforming BTPN, posited that in the digital era, banking is a secondary activity as people do not wake up wanting to bank. People want to eat, travel, shop, or build a home. Therefore, banking services must recede into the background, becoming enablers of these primary life activities.   

Bank Jago’s strategy is built on the premise that building a standalone digital bank is expensive and risky due to high Customer Acquisition Costs (CAC). Instead, Jago adopted a partnership-heavy model. The strategic alliance with Gojek (via GoPay) and later the GoTo ecosystem (Gojek, Tokopedia, GoTo Financial) became the cornerstone of this strategy.

By integrating deeply with Gojek, Jago gained access to millions of active users who were already transacting digitally but might lack a full bank account. This created a "Customer-Pull" dynamic: users were drawn to Jago not because of ad spend, but because it offered tangible utility within the apps they already used (e.g., no top-up fees for GoPay, seamless payments for GoFood). This structural advantage was solidified when Gojek (through GoPay) acquired a significant stake (about 22\%) in Bank Jago, aligning the incentives of the platform and the bank \cite{Haryanto2022}.

\subsection{Technological Changes}

In a decisive break from Indonesian banking tradition, Bank Jago became the first local bank to adopt a fully cloud-native core banking system. The bank selected Mambu as its SaaS (Software-as-a-Service) core banking engine and hosted it on the Google Cloud Platform (GCP) \cite{Mambu2021BankJago}.

This architectural choice provided three advantages over the legacy on-premise mainframes used by Bank Artos:

\begin{enumerate}
	\item Composable Architecture: Mambu allows Bank Jago to build banking products by assembling pre-built "blocks" via APIs. This means a new loan product or savings feature can be launched in weeks rather than months, a crucial capability for responding to ecosystem partner needs.  
	
	\item Elastic Scalability: Hosting on GCP using Google Kubernetes Engine (GKE) allows the bank’s infrastructure to auto-scale. During massive e-commerce events like "Harbolnas" (12.12 sales) on Tokopedia, transaction volumes can spike by orders of magnitude. A traditional legacy system would likely crash or require expensive over-provisioning. Jago’s cloud infrastructure expands automatically to handle the load and contracts afterwards, optimizing costs.  
	
	\item Cost Efficiency (OpEx vs. CapEx): By avoiding the purchase of physical data centers and servers, Jago shifted its cost structure from heavy Capital Expenditure to variable Operational Expenditure. This aligns costs with revenue. The bank pays for computing power only as it acquires customers, preventing the "cash burn" trap.   
\end{enumerate}

To execute the embedded finance strategy, Jago built a robust API (Application Programming Interface) layer. This allows external partners like Bibit and Gojek to "call" banking functions securely \cite{BankJago2021GoPayGojek}. For example, when a user buys a mutual fund on Bibit using Jago funds, Bibit’s app communicates directly with Jago’s core via API to execute the debit in real-time.  

Bank Jago also made an imporvement in it's security. Bank Jago implemented rigorous security standards, achieving ISO/IEC 27001:2013 certification for Information Security Management Systems. 

\subsection{Organizational Changes}

Bank Artos operated with a \cite{Savage1996} Generation 2/3 hierarchy: rigid, siloed, and slow. To support its digital ambitions, Bank Jago had to engineer a \cite{Savage1996} Generation 5 organization: networked, agile, and team-centric.


Bank Jago abandoned the traditional departmental structure (IT Dept, Marketing Dept, Product Dept) in favor of the "Spotify Model" \cite{jago2022integrated, jago2023integrated}. The workforce is organized into cross-functional "Squads," each focusing on a specific customer journey or product (e.g., the "Onboarding Squad" or "Investment Squad"). These squads are grouped into "Tribes". 

In this model, a squad contains all the necessary expertise: a developer, a product owner, a risk officer, and a compliance specialist all sitting together. This eliminates the "handover" delays typical of banks where a product team writes a spec and hands it over to IT. Squads have the autonomy to make decisions quickly.

\subsection{Financial Impact of Bank Jago Transformation}

Bank Artos relied almost exclusively on interest income from a small loan book. Bank Jago has diversified its revenue. While net interest income remains the primary driver (fueled by partnership lending), fee-based income from ecosystem transactions (e.g., investment fees, payment fees) has grown significantly \cite{jago2019annual,  jago2023integrated, jago2024annual}.

\begin{enumerate}
	\item User Growth: From a negligible base in 2019, Jago grew to 1.5 million users in 2021 , and skyrocketed to 15.3 million customers by the end of 2024, including 12.1 million funding customers.  
	
	\item Asset Growth: Total assets grew from IDR 733 billion in 2018 1 to IDR 28.5 trillion in 2024, a massive expansion funded by rights issues and deposit growth. 2   
\end{enumerate}

The efficiency gains from the cloud-native model and branchless strategy are evident in the financial ratios.

\begin{enumerate}
	\item BOPO: The ratio improved dramatically from the unsustainable 258\% in 2019 to a healthy level below 80\% in 2024. Bank Jago is now generating significant operational profit.  
	
	\item Net Profit: From a loss of IDR 165 billion in 2019, the bank turned profitable in 2021 (IDR 86 billion) and has continued to grow its bottom line, recording consistent profits in 2023 and 2024.   
\end{enumerate}


\subsection{IT Impact Analysis}

\begin{figure}[htbp]
	\centering
	\includegraphics[width=\linewidth]{it_impact_bank_jago.jpg}
	\caption{IT Impact Map of Bank Jago}
	\label{fig:it_impact_bank_jago}
\end{figure}

The transformation of Bank Jago can be understood as a staged progression across the four quadrants of the IT Impact Map, beginning from its origins as Bank Artos, a conventional bank operating with minimal digital capability from 1992 to 2019 as shown in Figure \ref{fig:it_impact_bank_jago}. This period reflects an Incremental Improvement stage in which technology played only a supportive, operational role. A fundamental shift occurred after the acquisition and rebranding, when Bank Jago began strategic investments in IT infrastructure and digital capability during 2019–2020, marking the transition into Business Process Reengineering \cite{Linggadjaya2022DigitalTransformation}. In this phase, the bank redesigned its internal processes by adopting cloud-native systems, microservices architecture, and standardized open APIs, thereby establishing a technological foundation that enabled modularity, scalability, and rapid innovation \cite{Khuan2022Competitiveness}.

With this technological backbone in place, Bank Jago advanced into the Emerging Opportunity quadrant by entering digital ecosystems and embedding its financial services within partner platforms. Its collaboration with major ecosystem players, including the GoTo group, significantly expanded its reach and accelerated customer acquisition through embedded finance models, proving that the bank’s digital architecture was capable of supporting new business opportunities and cross-platform integrations \cite{Linggadjaya2022DigitalTransformation}. This ecosystem-driven growth ultimately paved the way for Business Transformation, manifested through the introduction of the Life-Focused Solution (LFS). Under this strategic concept, Bank Jago reconceptualized its value proposition from offering banking products to delivering financial experiences integrated into customers’ daily life activities, enabled by continuous digital innovation and ecosystem connectivity \cite{Linggadjaya2022DigitalTransformation}.
Business Model of Bank Jago


\subsection{Business Model Analysis}

\begin{figure}[htbp]
	\centering
	\includegraphics[width=\linewidth]{business_model_bank_Jago.jpg}
	\caption{Business Model of Bank Jago}
	\label{fig:business_model_jago}
\end{figure}


The evolution of Bank Jago’s business model can be mapped across the four quadrants of the product–market expansion grid as shown in Figure \ref{fig:business_model_jago} , illustrating how the bank transitioned from a conventional institution into a digitally native financial services provider. In the Product–Market Enhancement quadrant, Bank Jago inherited the legacy position of Bank Artos, which operated as a traditional bank from 1992 to 2019 with a limited digital footprint and a narrow market reach. The acquisition and subsequent rebranding created the foundation for improving existing services while retaining the bank’s initial customer segments \cite{Linggadjaya2022DigitalTransformation}. Building on this base, Jago introduced Smart Pockets and Locked Pockets beginning in 2020, representing the Product Expansion quadrant. These innovations added new digital features to support personal financial management while continuing to serve its existing market. Such features were enabled by the bank’s redesigned backend systems and reflect the application of modular, API-driven architecture \cite{Khuan2022Competitiveness}.

The deployment of an Open API ecosystem and microservices architecture from 2020 onward marked a movement into the Market Expansion quadrant. This architectural shift allowed Bank Jago to broaden its reach to previously untapped digital customer segments by enabling seamless third-party integrations and positioning the bank as an interoperable financial service provider embedded across multiple digital channels \cite{Khuan2022Competitiveness}. Finally, the bank’s strategic partnership with the GoTo ecosystem beginning in 2021 pushed Jago into the Business Exploration quadrant. Through embedded finance integration with platforms such as Gojek, Tokopedia, and GoPay, Bank Jago expanded both its product use cases and market scope simultaneously, enabling new forms of digital engagement and accelerating customer acquisition through ecosystem-driven financial experiences \cite{Linggadjaya2022DigitalTransformation}.

\subsection{Comparative Analysis to Bank Jenius}


\begin{figure}[htbp]
	\centering
	\includegraphics[width=\linewidth]{business_model_jenius.jpg}
	\caption{Business Model of Bank Jenius}
	\label{fig:business_model_jenius}
\end{figure}

The business model evolution of Jenius can be explained using the product–market grid, which captures how the bank introduced digital innovations and expanded its strategic positioning through new products and new customer segments as shown in \ref{fig:business_model_jenius}. The grid illustrates how Jenius progressively moved from enhancing existing financial services toward exploring entirely new digital banking paradigms driven by technology.

At the Product–Market Enhancement quadrant (same products, same markets), Jenius focused on improving traditional banking activities through its mobile application. The ability to perform banking transactions entirely via smartphone, such as checking balances, transferring funds, and making payments, enhanced convenience and efficiency for existing customers. Jenius succeeded in simplifying basic financial tasks by shifting these services into a fully digital environment, making banking faster and more accessible \cite{8940837}. These improvements strengthened the value of existing products without altering the fundamental customer base.

In the Market Expansion quadrant (same products, new markets), Jenius leveraged its mobile-first approach to attract a new demographic segment: digitally savvy urban users. This group, characterized by mobile-oriented lifestyles and high technology adoption, had different expectations compared to traditional banking customers. The digital-only nature of Jenius, operating without conventional branches, aligned with the preferences of this new market segment. The paper highlights that Jenius successfully reached customers who were previously underserved or disengaged from traditional banks, positioning itself as a modern alternative tailored for millennials and digital natives \cite{8940837}.

The Product Expansion quadrant (new products, same markets) reflects Jenius’s introduction of innovative financial features that enhanced personal money management for its existing digital users. Products such as Dream Saver, Flexi Saver, Maxi Saver, and the x-Card extended Jenius's value proposition beyond basic banking. These features were designed to help customers automate savings, segment financial goals, and gain granular control over digital transactions. These innovations as key elements enabling users to “"manage users' finance wisely,” supporting Jenius’s commitment to customer-centric financial empowerment \cite{8940837}. These new products enriched the digital experience of the bank’s established market without requiring a shift in customer segments.

Finally, the Business Exploration quadrant (new products, new markets) represents Jenius’s most transformative initiatives, such as the adoption of human-centered design and the launch of the Jenius Co.Create community platform. These initiatives allow customers to participate in co-creating financial solutions, forming a collaborative ecosystem that goes beyond traditional banking boundaries. The paper positions Jenius within the larger “Bank 4.0” movement, where banking becomes embedded in everyday life and delivered entirely through digital touchpoints \cite{8940837}. This exploration into new digital paradigms and community-driven innovation reflects Jenius’s ambition to redefine how financial services are conceived, developed, and experienced.


\subsection{Comparative Analyisis to KakaoBank}

\begin{figure}[htbp]
	\centering
	\includegraphics[width=\linewidth]{it_impact_kakao.jpg}
	\caption{IT Impact of KakaoBank}
	\label{fig:it_impact_kakao}
\end{figure}

As shown in Figure  \ref{fig:it_impact_kakao}, in the Incremental Improvement quadrant, KakaoBank enhanced basic banking convenience by applying digital technology to simplify user processes. The paper explains that KakaoBank enabled remote ID verification using facial recognition, which made account opening faster and eliminated the need for physical visits. In addition, KakaoBank provided a 24/7 mobile-based service, giving customers continuous access to banking functions \cite{choi2020digital}.

In the Business Process Design/Reengineering quadrant, KakaoBank adopted a new operational model through the establishment of branchless banks (internet-only banks). The paper states that Korean regulations allowed banks without physical branches to operate, and KakaoBank utilized this opportunity to conduct all processes digitally. Without physical branches, internal processes such as identity verification, customer service, and transactions were redesigned to function entirely in a digital environment. This reflects process reengineering driven by technology \cite{choi2020digital}.

In the Emerging Opportunity quadrant, the digital capabilities already established enabled KakaoBank to access new opportunities that traditional models could not support. The paper notes that KakaoBank experienced rapid customer acquisition through platform integration, particularly due to its connection with a platform ecosystem that already had a large user base. This integration allowed KakaoBank to attract a significant number of customers in a short period and illustrates how digital capabilities open access to broader market segments \cite{choi2020digital}.

In the Business Transformation quadrant, the paper indicates that KakaoBank contributed to changes in the banking industry in Korea through the adoption of a digital-only model. The emergence of internet-only banks such as KakaoBank influenced the national banking landscape, including how banking services were delivered and accessed. The digital model adopted by KakaoBank became part of a larger industry transformation described in the report \cite{choi2020digital}.

\begin{figure}[htbp]
	\centering
	\includegraphics[width=\linewidth]{business_model_kakao.jpg}
	\caption{Business Model of KakaoBank}
	\label{fig:business_model_kakao}
\end{figure}



As shown in Figure \ref{fig:business_model_kakao}, in the Product–Market Enhancements quadrant (Same Product, Same Market), KakaoBank improved the efficiency and accessibility of existing banking services by introducing remote ID verification. The paper explains that KakaoBank enabled customers to open accounts remotely through digital identity verification, which simplified the account-opening process and removed the need for in-person interaction. This enhancement streamlined access to traditional banking services without changing the product itself or the target market \cite{choi2020digital}.

In the Market Expansion quadrant (Same Product, New Market), KakaoBank leveraged its connection to the KakaoTalk platform, used by more than 40 million users, to extend the reach of its existing banking services to a broader digital audience. According to the paper, KakaoBank’s integration with KakaoTalk allowed it to rapidly acquire new customers by tapping into the platform’s large established user base. This represents market expansion because the products remained the same, but were delivered to a significantly larger and previously untapped digital user segment \cite{choi2020digital}.

In the Product Expansion quadrant (New Product, Same Market), KakaoBank introduced new savings products targeted at its existing customer base, including Safebox and the 26-weeks installment savings option. The report highlights these offerings as innovative deposit and savings products designed to appeal to users already within the KakaoBank ecosystem \cite{choi2020digital}.

In the Business Exploration quadrant (New Product, New Market), KakaoBank entered new financial business areas by offering a securities brokerage service. As described in the paper, KakaoBank expanded beyond its existing product categories and moved into the securities market, representing both a new product line and a new customer segment. This shift reflects exploratory activities that extend the bank’s business model beyond conventional retail banking \cite{choi2020digital}.


\section{Comparative Synthesis}

To understand the unique position of Bank Jago within the digital banking landscape, it is necessary to synthesize the findings by comparing it with Bank Jenius (Indonesia) and KakaoBank (South Korea). While all three entities utilized IT to move into the "Business Exploration" quadrant of the product-market grid, their underlying strategic drivers and architectural approaches differ significantly.

\subsection{Strategic Positioning: Standalone vs. Embedded}

The primary distinction lies in customer acquisition strategy. 
\begin{itemize}
	\item \textbf{Bank Jenius} adopts a "Lifestyle Banking" approach. It operates primarily as a B2C (Business to Consumer) standalone app, relying on product innovation (e.g., Flexi Saver, foreign currency cards) to attract users directly to its platform.
	\item \textbf{KakaoBank} utilizes a "Platform Leverage" strategy. It was born out of a dominant social messaging platform (KakaoTalk). It converts existing social users into banking users by minimizing friction in the user interface.
	\item \textbf{Bank Jago} pursues an "Embedded Finance" strategy. Unlike Jenius, Jago does not rely solely on its own app as the destination. Instead, it embeds its services into other platforms (Gojek, Bibit, Stockbit). Jago functions as the "financial utility" layer behind these ecosystems.
\end{itemize}

\subsection{Infrastructure and Agility}

From an IT perspective, Bank Jago's transformation is the most radical regarding infrastructure. Bank Jenius was incubated within BTPN, initially relying on a hybrid of legacy and new systems. In contrast, Bank Jago (formerly Artos) completely discarded its legacy core to adopt a cloud-native SaaS core (Mambu) hosted on Google Cloud. This allows for a higher degree of composability compared to traditional digital banking arms.

\begin{table}[htbp]
	\caption{Strategic Comparison of Jago, Jenius, and KakaoBank}
	\label{tab:comparison_synthesis}
	\begin{tabular}{|p{2cm}|p{1.8cm}|p{1.8cm}|p{1.8cm}|}
		\hline
		\textbf{Dimension} & \textbf{Bank Jago} & \textbf{Bank Jenius} & \textbf{KakaoBank} \\ \hline
		\textbf{Origin} & Transformation of Legacy Bank (Artos) & Internal Digital Spinoff (BTPN) & Internet-Only Bank License \\ \hline
		\textbf{Primary Channel} & Embedded in Ecosystems (Gojek/Bibit) & Standalone Mobile App & Integrated via Social Platform (KakaoTalk) \\ \hline
		\textbf{IT Architecture} & Cloud-Native (Mambu/GCP) & Hybrid Digital Core & Mobile-First Proprietary Stack \\ \hline
		\textbf{Value Proposition} & Life-Centric (Invisible Banking) & Life-Finance Management & Social \& Convenient Banking \\ \hline
		\textbf{Business Model Focus} & B2B2C (Partnership) & B2C (Direct) & B2C (Platform) \\ \hline
	\end{tabular}
\end{table}

Table \ref{tab:comparison_synthesis} summarizes these differences. While Jenius and KakaoBank focus on optimizing the banking experience for the user, Bank Jago focuses on integrating banking into the user's life context. This confirms that Bank Jago's transformation from Bank Artos was not merely a digitization of services, but a fundamental shift toward a Savage Generation 5 networked organization that competes on connectivity rather than just product features.

\section{Conclusion}

The transformation of Bank Artos into Bank Jago is a definitive case study in the digitization of emerging market finance. It demonstrates that Information Technology is not merely a support tool but the primary driver of strategic renewal.

\begin{enumerate}
	\item     Strategy: The pivot from a "Pipeline" model (Bank Artos) to a "Platform/Ecosystem" model (Bank Jago) unlocked exponential growth (from 50 thousand to 15 million customers) that was structurally impossible under the legacy model.
	
	\item Capabilities: The adoption of cloud-native infrastructure (Mambu/GCP) and agile organizational structures provided the speed and scalability required to execute the ecosystem strategy. This resolved the "organizational lag" that haunted Bank Artos.
	
	\item Value: The transformation corrected the fundamental insolvency of the legacy bank. By leveraging ecosystems to lower CAC and cloud tech to lower OpEx, Bank Jago converted a loss-making institution (BOPO 258\%) into a highly profitable and efficient digital bank (NPL 0.2\%).
\end{enumerate}

The transformation of Bank Artos into Bank Jago is a prime case study in the digitization of emerging market finance. It demonstrates that Information Technology is not merely a support tool but the primary driver of strategic renewal.

% ---------- Bibliography ----------
\bibliographystyle{IEEEtran}
\bibliography{references}  % references.biba must be in same folder

\end{document}
