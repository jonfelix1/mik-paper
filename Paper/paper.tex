\documentclass[conference]{IEEEtran}
\IEEEoverridecommandlockouts

\usepackage{cite}               % IEEE numeric citations
\usepackage{amsmath,amssymb,amsfonts}
\usepackage{algorithmic}
\usepackage{graphicx}
\usepackage{textcomp}
\usepackage{xcolor}
\usepackage{tikz}
\usetikzlibrary{positioning, fit, arrows.meta, shapes}
\usetikzlibrary{shapes.geometric, arrows, positioning}

\newcommand{\empt}[2]{$#1^{\langle #2 \rangle}$}



\def\BibTeX{{\rm B\kern-.05em{\sc i\kern-.025em b}\kern-.08em
    T\kern-.1667em\lower.7ex\hbox{E}\kern-.125emX}}

\begin{document}

\title{Business Model Transformation of Bank Artos to Bank Jago  \\
\thanks{Identify applicable funding agency here. If none, delete this.}
}

\author{\IEEEauthorblockN{1\textsuperscript{st} Samuel Mualatua Jeremy N}
	\IEEEauthorblockA{\textit{dept. name of organization (of Aff.)} \\
		\textit{name of organization (of Aff.)}\\
		City, Country \\
		email address or ORCID}
	\and
    \IEEEauthorblockN{2\textsuperscript{nd} Jose Galbraith Hasintongan}
	\IEEEauthorblockA{\textit{dept. name of organization (of Aff.)} \\
		\textit{name of organization (of Aff.)}\\
		City, Country \\
		email address or ORCID}
	\and
	\IEEEauthorblockN{3\textsuperscript{rd} Muhammad Firdaus}
	\IEEEauthorblockA{\textit{dept. name of organization (of Aff.)} \\
		\textit{name of organization (of Aff.)}\\
		City, Country \\
		email address or ORCID}
	\and
	\IEEEauthorblockN{4\textsuperscript{th} Jon Felix Germinian}
	\IEEEauthorblockA{\textit{dept. name of organization (of Aff.)} \\
		\textit{name of organization (of Aff.)}\\
		City, Country \\
		email address or ORCID}
}


\maketitle

\begin{abstract}
This paper examines how information technology (IT) has transformed the business model of PT Bank Jago Tbk, which was previously known as Bank Artos. Following its acquisition, the former Bank Artos was redesigned and relaunched as Bank Jago, evolving from a small traditional bank into a fully digital, ecosystem-based bank. Using a descriptive qualitative approach, this study analyzes how IT has changed the bank’s value creation, customer acquisition, and revenue structure. The paper also compares Bank Jago with similar companies in Indonesia and abroad, including ******** The findings indicate that IT has become the core strategic enabler of Bank Jago’s transformation, shifting its business positioning from a conventional bank to a Inovation challenger in Indonesia’s digital banking landscape. This study contributes to a better understanding of how legacy financial institutions can leverage technology to restructure their business models and remain competitive in the era of digital disruption.
\end{abstract}

\begin{IEEEkeywords}
Bank Jago, Bank Artos, digital transformation, business model
\end{IEEEkeywords}

\section{Introduction}

Digital transformation has become a necessity in the banking sector due to rising public expectations for services that are fast, efficient, safe, and secure, as well as financing and services that can be conducted at any location \cite{Linggadjaya2022DigitalTransformation}.

Bank Jago was originally established as Bank Artos, a conventional bank created in 1992 and later listed on the stock exchange in 2016. A major turning point occurred in 2019 when Bank Artos was acquired by Metamorfosis Ekosistem Indonesia and Wealth Track Technology, becoming its new controlling shareholders. Following this acquisition, on June 11, 2020, the company name was officially changed to Bank Jago.

The rise of digital competition in Indonesia further strengthens the relevance of examining Bank Jago’s transformation. Advances in financial technology have intensified industry rivalry, with fintech firms offering more convenience, lower costs, and broader access than traditional banks \cite{Khuan2022Competitiveness}. Digital banking transactions in Indonesia continue to increase each year, reflecting a nationwide shift toward simpler and faster digital financial services \cite{Khuan2022Competitiveness}. Within this landscape, Bank Jago represents a significant example of deep strategic transformation, having made substantial investments in technology and infrastructure to reinvent itself as a fully digital, ecosystem-based bank \cite{Khuan2022Competitiveness}. This makes Bank Jago a relevant case for understanding how a legacy institution can reposition its business model in response to digital disruption.

\section{Literature Review}

\subsection{Business Model Framework}

According to \cite{Applegate2009} a business model is a complex system comprising three interlocking components of strategy, capability, and value. This framework serves as the primary tool for this case study to categorize the changes driven by IT.

\begin{enumerate}
	\item Strategy: This component defines the organization's market positioning, product positioning, and business network positioning. In the context of digital banking, IT enables a shift from "product-push" strategies to "customer-pull" strategies, allowing for personalization and the servicing of previously unprofitable segments (e.g., the unbanked) through low-cost digital channels.
	\item Capabilities: This refers to the resources (people, processes, and infrastructure) required to execute the strategy. \cite{Applegate2009} emphasizes that IT is no longer just support; it is the infrastructure itself. For Bank Jago, this involves the transition from legacy mainframes to cloud-native architectures (Mambu/Google Cloud), which fundamentally alters the cost structure from Capital Expenditure (CapEx) to Operational Expenditure (OpEx).
	\item Value: This component analyzes the economic logic of how value is created for stakeholders. IT changes the "profit formula" by reducing transaction costs and Customer Acquisition Costs (CAC) through ecosystem integration rather than physical branch expansion. 
\end{enumerate}

\subsection{Digital Transformation in Banking }

The global banking sector is undergoing a shift in the digital era. The advent of information and communication technology has forced a departure from traditional, branch-based banking models toward integrated, digital-first ecosystems \cite{osei2023unlocking}. This digital shift affects all aspects of the baking industry, from back-office operations and risk management to customer engagement and service delivery.

The banking industry, once a bastion of tradition, now faces disruptive pressures from multiple fronts:

\begin{enumerate}
    \item Evolving Customer Expectations: Customers of the modern era now demand faster, easier, and more connected banking experiences through their smartphones and online platforms
    \item The FinTech Challenge: FinTech firms and neobanks, leverage agile technology and customer-centric models. Nearly 90\% of banks fear losing business to these agile competitors \cite{osei2023unlocking}. These FinTech companies and new digital banks have become significant rivals in areas like digital payments and peer-to-peer lending \cite{alqararah2025role}
\end{enumerate}

For traditional banks, digital transformation is an essential journey to stay competitive in the industry \cite{alqararah2025role}. This change isn't just about making things digital; it's about completely changing how the bank operates and how it makes money \cite{alqararah2025role}.


\subsection{The Ecosystem Driver and Embedded Finance}

Complementing \cite{Applegate2009}, this research integrates the concept of Embedded Finance and the Ecosystem Driver model described by \cite{Weill2018}. In this model, banks no longer act as standalone destinations but embed their financial services (payments, lending, savings) directly into the customer’s daily digital activities (e.g., ride-hailing, e-commerce).  

This aligns with the shift from "Pipeline" business models (linear value creation) to "Platform" business models (networked value creation). Research by \cite{Linggadjaya2022DigitalTransformation} specifically on Bank Jago highlights that digital transformation requires a "digital culture" and "collaboration" capabilities, which allows the bank to leverage external data for credit scoring, a key differentiator from conventional models.   

\subsection{The Evolutionary Trajectory of Management Generations}
\cite{Savage1996} framework divides the progression of organizational maturity to five distinct generations. Each generation represents a fundamental shift in how people, technology, and resources are combined to create value. The critical theoretical argument is that while technology has advanced to the Fifth Generation (parallel processing/networking), most organizations remain trapped in Second or Third Generation management structures (hierarchies/matrix), creating a "organizational lag" that stifles digital transformation.

\begin{itemize}
	\item Generation 1: Proprietorship (The Craft Era).
	Characterized by the owner-entrepreneur who performs all functions. Information flow is informal and contained within the mind of the proprietor, limiting scalability.
	
	\item Generation 2: Steep Hierarchies (The Industrial Era).
	Based on the division of labor and "Command and Control" logic. Functional departments (e.g., Finance, Ops) operate in silos with vertical information flow, prioritizing stability over agility.
	
	\item Generation 3: The Matrix (Electronic Data Processing)
	The introduction of IT to automate manual tasks within silos. While data is digitized, it creates "Islands of Automation" where systems do not communicate. This represents the legacy Bank Artos model: rigid and bureaucratically fragmented.
	
	\item Generation 4: Computer Interfacing (Technical Integration).
	Organizations establish technical connectivity (systems talking to systems) but retain cultural silos. The network exists, but the management style remains hierarchical.
	
	\item Generation 5: Human Networking (The Knowledge Era).
	Gen 5 (Knowledge/Teaming): Characterized by cross-functional agility and real-time connectivity. Bank Jago’s adoption of "Life-Centric" financial solutions and agile "Squad" organizational structures represents a move toward Generation 5, where IT enables the organization to function as a dynamic network of partners rather than a rigid hierarchy.   
\end{itemize}



\section{Methodology}

\subsection{Research Design}

This study employs a qualitative descriptive research design using a single-case study approach. The case study method is selected because it allows for an in-depth investigation of a contemporary phenomenon (digital banking transformation) within its real-life context of Bank Jago. This approach is particularly suitable for answering "how" and "why" of IT changes the business model of Bank Jago. 

\subsection{Data Collection}

Data was collected using secondary data analysis techniques. To ensure the validity and reliability of the research, data was triangulated from multiple official and academic sources covering the period 2018 (Pre-acquisition/Bank Artos) to 2024 (Post-transformation/Bank Jago):


\begin{enumerate}
	\item Corporate Filings: Annual Reports (2019–2023), Sustainability Reports, and Financial Statements published on the Indonesia Stock Exchange (IDX) and the Bank Jago investor relations website. Key focus areas included the Management Discussion and Analysis (MD\&A) sections to extract strategic intent.
	
	\item Technology Case Studies: Technical documentation and case studies published by Bank Jago’s technology partners, specifically Google Cloud Platform (GCP) and Mambu. These documents provide the technical evidence of infrastructure changes (e.g., cloud migration, API architecture).
	
	\item Academic Literature: Peer-reviewed journal articles discussing Bank Jago’s competitiveness, digital transformation strategies, and governance challenges ,.   
\end{enumerate}


\subsection{Data Analysis Technique}

The collected data was analyzed using content analysis and comparative analysis based on the \cite{Applegate2009} Framework components:

\begin{enumerate}
	\item     Strategic Comparison: Contrasting Bank Artos’s 2018 market positioning (conventional, branch-heavy) with Bank Jago’s 2024 ecosystem positioning (embedded, branchless).
	
	\item Capability Assessment: Mapping the shift in IT infrastructure (Legacy Core vs. Cloud Native) and Organizational Structure (Functional vs. Agile Squads).
	
	\item Economic Logic Evaluation: Comparing key financial ratios—specifically Cost-to-Income Ratio (CIR), Net Interest Margin (NIM), and Customer Acquisition Cost (CAC)—to determine the economic impact of the business model change.
\end{enumerate}

% ---------- Bibliography ----------
\bibliographystyle{IEEEtran}
\bibliography{references}  % references.biba must be in same folder

\end{document}
